There is an increasing interest in \emph{Native Language Identification} (NLI).
NLI is a task of detecting the native language (L1) of the author of the text.
This may be treated as a sub-task of \emph{author profiling}, where the goal is to predict traits of the author such as gender, age or psychometric traits \citep{estival2007author}.
Author profiling is used, for instance, to narrow down the suspects of criminal activities\citep{abbasi2005applying} by filtering them with the attributes detected from the communication on online mediums.
A possible commercial application is to collect customer information from the reviews for market intelligence\citep{glance2005deriving}, which could be especially effective when used in conjunction with sentiment analysis.

On the top of these applications, NLI can be used to create a better writing tutor system a non-native speaker of English.
When using a non-native language (L2), one often applies the knowledge of their L1.
This well-studied phenomenon is called \emph{L1 transfer}.\color{red}CITE\color{black}
L1 transfer could result to correct expressions, but could also lead to errors which are specific to the particular L1. 
Based on this observation, there are several works \citep{chang2008automatic, rozovskaya2010generating, rozovskaya2011algorithm, dahlmeier2011correcting} which investigate how the knowledge of the authors' first language could improve the grammatical error detection and correction in their texts.
A writing tutor system can first used NLI, and use the detected L1 to have an improved error detection and correction.

There have been many works on NLI.
Early works in NLI\citep{koppel2005determining, tsur2007using} showed the effectiveness of Support Vector Machine (SVM) in this task.
SVM continues to be the popular choice, and 13 out of 24 participating teams, including the winning team \citep{jarvis2013maximizing} in the Native Language Identification Shared Task \citep{tetreault2013report} utilised SVM.
The difference among these works comes down to the feature sets used, which range from lexical n-grams \citep{koppel2005determining, tsur2007using, jarvis2013maximizing} to fragments of Tree Substitutional Grammar \citep{swanson2012native}.
In other words, feature engineering has been a large focus of NLI.

In this report, we perform NLI on the subset of Cambridge Learner Corpus \citep{nicholls2003cambridge, yannakoudakis2011new} using a character-based Convolutional Neural Network (CNN).
Specifically, we use the character-based CNN designed by \cite{zhang2015character} to perform various classification tasks.
We chose to use a character-based CNN as opposed a word-based CNN, as lexical features are in general considered to be more appropriate for the task of topic categorisation than of author profiling \citep{kochmar2011identification}.
CNNs have gained in popularity for its success in various practical applications, such as image recognition \color{red} cite \color{black}, natural language processing  \color{red} cite \color{black} and playing Go \color{red} cite \color{black}.
CNNs have a great advantage over traditional machine learning techniques such as SVM, which is the lack of feature engineering.
The downside is that they typically require a large-scale dataset, and the dataset we have is fairly small.
Indeed, applying the implementation by \cite{zhang2015character} with no modification to the dataset does not perform better than a simple baseline system which assigns the label arbitrarily.
Another potential explanation for the poor performance is the length of each document in the dataset.
\cite{zhang2015character} uses the first 1014 characters of each document as an input.
However, many of the documents are shorter than 1014 characters and therefore require padding.
This may have introduced some noise to the input.
To partially mitigate these problems, we use a smaller window and sample a section of the document randomly, rather than always taking the first part.
The results show that CNN performs better with a small window size.
However, the performance still does not match that of SVM.

The remaining of the report proceeds as follows; \color{red} FINISH \color{black}