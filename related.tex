The first work on NLI that we are aware of is that of \cite{koppel2005determining}.
They used SVM to perform five-class classification between Russian, Czech, Bulgarian, French, and Spanish.
The features they extracted include stylistic features such as function words, character n-grams, and grammatical errors.
They report the accuracy of 80.2\%.
\cite{tsur2007using} performs the exact same task using SVM and investigates the effect of each features used.
\cite{wong2011exploiting} experiment with substructures of a parse tree as additional features and achieve the accuracy of 80\% in seven-class classification with SVM.
\cite{swanson2012native} also used SVM and fragments of Tree Substitutional Grammar as additional features. This gave 78.4\% accuracy in the same task as \cite{wong2011exploiting}.
\cite{swanson2012native} replicate the work by \cite{wong2011exploiting} and report the accuracy of 72.6\%.

Aside from SVM, Latent Dirichlet Analysis was applied to NLI with seven languages by \cite{dras2011topic}.
This gave the accuracy of 56.9 \%, which was lower than 64.1\% accuracy that their baseline SVM gave.
All the works mentioned up to this point uses the ICLE corpus \citep{granger2002international}.

In 2013, the First Native Language Identification Shared Task was held \citep{tetreault2013report}.
The task was to perform NLI on the new TOEFL11 corpus \citep{tetreault2013report} which included essays written by native speakers of 11 languages.

As mentioned previously, the majority, including the winning team  \citep{jarvis2013maximizing}, utilised SVM.
\cite{jarvis2013maximizing} achieves the accuracy of 83.6\%.

The other approaches seen in the shared task include MaxEnt, Ensemble, and Discriminant Function Analysis \citep{tetreault2013report}.

